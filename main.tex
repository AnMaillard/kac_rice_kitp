\documentclass[amsmath,amssymb,nofootinbib,prd]{article}
%\pdfoutput=1

%,superscriptaddress

\usepackage{graphicx}
\usepackage{float}
\usepackage{fullpage}
\usepackage[colorlinks=true,citecolor=blue]{hyperref}
\usepackage{dsfont}
\usepackage[utf8]{inputenc}
\usepackage[english]{babel}
\usepackage{mathtools}
\usepackage{amsmath}
\usepackage{bm}
\usepackage{braket}
\usepackage{latexsym}
\usepackage{amssymb}
\usepackage[compat=1.1.0]{tikz-feynman}
\usepackage{comment}



% new commands
\newcommand*\NewPage{\null\thispagestyle{empty}\newpage}
\DeclareUnicodeCharacter{00A0}{~}
\def \L{{\cal L}}
\def \({\left(}
\def \){\right)}
\def \[{\left[}
\def \]{\right]}
\newcommand{\tbf}[1]{{\textbf{#1}}}
\newcommand{\bsy}[1]{{\boldsymbol{#1}}}
\newcommand{\txt}[1]{\text{#1}}
\newcommand{\defeq}{\vcentcolon=}
\newcommand{\eqdef}{=\vcentcolon}
\newcommand{\cv}[1]{\underline{#1}}
\newcommand{\bQ}{{\textbf {Q}}}
\newcommand{\bq}{{\textbf {q}}}
% \newcommand{\bm}{{\textbf {m}}}
\newcommand{\br}{{\textbf {r}}}
\newcommand{\bU}{{\textbf {U}}}
\newcommand{\bu}{{\textbf {u}}}
\newcommand{\bV}{{\textbf {V}}}
\newcommand{\bF}{{\textbf {F}}}
\newcommand{\bY}{{\textbf {Y}}}
\newcommand{\bff}{{\textbf {f}}}
\newcommand{\bG}{{\textbf {G}}}
\newcommand{\bW}{{\textbf {W}}}
\newcommand{\bZ}{{\textbf {Z}}}
\newcommand{\bh}{{\textbf {h}}}
\newcommand{\bw}{{\textbf {w}}}
\newcommand{\bv}{{\textbf {v}}}
\newcommand{\bA}{{\textbf {A}}}
\newcommand{\bB}{{\textbf {B}}}
\newcommand{\bhx}{\hat {\textbf {x}}}
\newcommand{\bX}{{\textbf {X}}}
\newcommand{\bx}{{\textbf {x}}}
\newcommand{\bo}{{\textbf {o}}}
\newcommand{\blambda}{{\boldsymbol{\Lambda}}}
\newcommand{\btau}{{\boldsymbol{\tau}}}
\newcommand{\by}{{\textbf {y}}}
\newcommand{\bz}{{\textbf {z}}}
\newcommand{\bl}{{\textbf {l}}}
\newcommand{\bs}{{\textbf {s}}}
\newcommand{\bS}{{\textbf {S}}}
\newcommand{\bc}{{\textbf {c}}}
\newcommand{\bk}{{\textbf {k}}}
\newcommand{\bR}{{\textbf {R}}}
\newcommand{\beps}{{\boldsymbol {\epsilon}}}
\newcommand{\bxigma}{{\boldsymbol{\Sigma}}}
\newcommand{\bxi}{{\boldsymbol{\xi}}}
\newcommand{\bzero}{{\textbf{0}}}
\newcommand{\ba}{{\textbf {a}}}
\newcommand{\cC}{{\mathcal{C}}}
\newcommand{\cB}{{\mathcal{B}}}
\newcommand{\lab}[1]{\label{#1}}
\newcommand{\bxii}{{\boldsymbol{\xi}}}
\newcommand{\bre}{{\textbf {e}}}
\newcommand{\bytilde}{{\tilde{\textbf {y}}}}
\newcommand{\bxhat}{{\tilde{\textbf {x}}}}
\newcommand{\mc}{\mathcal}
\newcommand{\eps}{\varepsilon}
\newcommand{\calF}{\mathcal F}
\newcommand{\s}{\sigma}
\renewcommand{\d}{\text{d}}
\newcommand{\e}{\text {e}}
\newcommand{\ds}{\Delta\sigma}
\newcommand{\hw}{h^{\rm W}}
\newcommand{\sw}{\sigma^{\rm W}}
\newcommand{\<}{\langle}
\renewcommand{\>}{\rangle}
\newcommand{\de}{\partial}
\newcommand{\la}{\langle}
\newcommand{\ra}{\rangle}
\newcommand{\eq}{\text{ eq}}
\newcommand{\sign}{\text{ sign}}
\newcommand{\ua}{\uparrow}
\newcommand{\da}{\downarrow}
\newcommand{\be}{\begin{equation}}
\newcommand{\ee}{\end{equation}}
% \newcommand{\bea}{\begin{eqnarray}}
% \newcommand{\eea}{\end{eqnarray}}
\newcommand{\mE}{\mathbb{E}}
\newcommand\smallO{
  \mathchoice
    {{\scriptstyle\mathcal{O}}}% \displaystyle
    {{\scriptstyle\mathcal{O}}}% \textstyle
    {{\scriptscriptstyle\mathcal{O}}}% \scriptstyle
    {\scalebox{.7}{$\scriptscriptstyle\mathcal{O}$}}%\scriptscriptstyle
  }
\newcommand{\bea}{\begin{align}}
\newcommand{\eea}{\end{align}}

\newtheorem{theorem}{Theorem}[section]
\newtheorem{lemma}[theorem]{\textbf{Lemma}}
\newtheorem{thm}[theorem]{\textbf{Theorem}}
\newtheorem{remark}[theorem]{\textbf{Remark}}
\newtheorem{proposition}[theorem]{\textbf{Proposition}}
\newtheorem{corollary}[theorem]{\textbf{Corollary}}
\newtheorem{definition}[theorem]{\textbf{Definition}}


\DeclareMathOperator{\atanh}{atanh}
\DeclareMathAlphabet{\varmathbb}{U}{bbold}{m}{n}
\newcommand{\id}{\mathds{1}}
\newcommand{\EE}{\mathbb{E}}
\newcommand{\noi}{\noindent}
\renewcommand{\d}{{\rm d}}
\renewcommand{\P}{{\rm P}}
% \newcommand{\eq}[1]{\begin{align}#1\end{align}}
\newcommand{\B}{{\rm B}}
\newcommand{\Tr}{{\rm Tr}}
\newcommand{\iif}{\Longleftrightarrow}
\newcommand{\bbR}{\mathbb{R}}
% \newcommand{\bF}{{\bf{F}}}
% \newcommand{\bx}{{\bf{x}}}
%

\begin{document}

\title{The Kac-Rice formula : basic definitions and a first application}
\date{\today}
\author{Antoine Maillard}
\maketitle
%
\renewcommand{\labelitemi}{$\bullet$}


\section{The Kac-Rice formula}

\subsection{The area formula}
	The Kac-Rice formula is intuitively not derived from any involved probabilistic tool, but rather is a consequence of a purely geometric result, called the area formula (itself a consequence of the more general corea formula), described in much more generality by Federer \cite{federer1959curvature}, and stated for instance in \cite{azais2009level}. This formula is the generalization of the following non-rigorous intuition: for a smooth function $f : \bbR \to \bbR$, and $T \subset \bbR$, denoting $N_f(u,T)$ the number of solutions to the equation $f(t) = u$ with $t \in T$, one would want to write informally:
	\begin{align}
	N_f(u,T) &= \int_{f(T)} \delta\left(v - u\right) \mathrm{d}v, \\
	  &= \int_T \delta\left(f(t) - u\right) |f'(t)| \mathrm{d}t.
	\end{align}
	The area formula generalizes and makes rigorous this intuition, by showing a weak version of this last equality. We follow here the statement of \cite{azais2009level}. 
	\begin{proposition}[Area formula]\label{prop:area}
	Let $f : U \to \bbR^d$ be a $\mathcal{C}^1$ function defined on an open subset $U$ of $\bbR^d$. Assume that the sets of critical values of $f$ has zero Lebesgue measure, and denote $N_f(u,T)$  the number of solutions to the equation $f(t) = u$ with $t \in T$. Then, for any Borel set $T \subset \bbR^d$, and any $g : \bbR^d \to \bbR$ continuous and bounded:
	\begin{align}
\int_{\bbR^d} g(u) N_f(u,T) &= \int_T \, \left|\det f'(t)\right| \, g(f(t)) \,\mathrm{d}t. 	
	\end{align}
	\end{proposition}
	
	
	\subsection{Informal derivation of Kac-Rice}
	
	Consider a compact manifold $\mathcal{M}$ of dimension $n$ (think of the unit sphere in $n$ dimensions), equipped with a measure $\mu_{\mathcal{M}}$, and a random function $f : \mathcal{M} \to \bbR$. We want to use the area formula Prop.~\ref{prop:area} to estimate the moments of the number of critical points of $f$. Given the hypotheses of Prop.~\ref{prop:area}, a reasonable hypothesis is to assume that $f$ is almost surely a \emph{Morse} function, i.e. that \emph{all its critical points are non-degenerate}. Since $\mathcal{M}$ is compact, one easily deduces that the number of critical points of a Morse function is finite \footnote{Note that the numbers of critical points of different indices of a Morse function are constrained by the topology of $\mathcal{M}$ by the Morse inequalities, see \cite{milnor1963morse} for a review on Morse theory.}. For any $k \in \mathbb{N}$ and Borel set $B \subseteq \bbR$, we define $\mathrm{Crit}_{f,k}(B)$ to be the number of critical points $x \in \mathcal{M}$ of $f$ such that $f(x) \in B$ and such that the index of $\mathrm{Hess}\, f(x)$ is equal to $k$. The informal area formula would read:
	\begin{align}
	\mathrm{Crit}_{f,k}(B) &= \int_{\mathcal{M}} \mathrm{\mu}_{\mathcal{M}}(\mathrm{d}x) \delta\left(\mathrm{grad}\, f(x)\right) \left|\det \, \mathrm{Hess}\, f(x)\right| \, \mathds{1}\left[f(x) \in B,\, \mathrm{i}\left(\mathrm{Hess}\, f(x)\right) = k\right]
	\end{align}
	
	Taking the expectation of this equality, one directly obtains the Kac-Rice formula:
	\begin{proposition}[Kac-Rice formula, informal]\label{prop:KR}Denote  $\varphi_{\mathrm{grad}\, f(x)}(0)$ the density of $\mathrm{grad}\, f(x)$ with respect to the Lebesgue measure on $\bbR^{n-1}$, taken at $0$. Then:
	\begin{align*}
	\EE \, \mathrm{Crit}_{f,k}(B) &= \int_{\mathcal{M}} \mathrm{\mu}_{\mathcal{M}}(\mathrm{d}x) \EE \left[\left|\det \, \mathrm{Hess}\, f(x)\right| \, \mathds{1}\left[f(x) \in B,\, \mathrm{i}\left(\mathrm{Hess}\, f(x)\right) = k\right] \middle| \mathrm{grad}\, f(x) = 0 \right] \varphi_{\mathrm{grad}\, f(x)}(0).
	\end{align*}
	\end{proposition}\vspace{0.cm}
	
	\paragraph{Remarks}:
	\begin{enumerate}
	\item The rigorous derivation of this formula is much more involved, as one has to start from the weak equality of Prop.~\ref{prop:area} and to use continuity arguments in order to obtain an equality at $u = 0$, see the proof of Kac-Rice performed in \cite{azais2009level}. This is the first reason for which the formula is mainly stated for Gaussian random fields, since these continuity arguments are then easier to establish. The second is that in general, conditional expectations of non-Gaussian random variables are intractable, making the Kac-Rice formula effectively useless. Under many technical conditions, one can however derive rigorous non-Gaussian versions of the Kac-Rice formula, stated for example as Thm.12.1.1 of \cite{adler2009random} or Thm.6.7 of \cite{azais2009level}. 
	\item One of the difficulties of evaluating the right-hand side of the Kac-Rice formula comes from the model of random matrices that arise from the distribution of the Hessian conditioned by the gradient being $0$. Even for Gaussian random fields, this is in general a heavily correlated Gaussian random matrix, for which very few results exist. 
	\item The Kac-Rice formula can be generalized to compute higher moments of the variable $\mathrm{Crit}_{f,k}(B)$ as well (Thm 6.3 of \cite{azais2009level}), and can therefore be used to compute the second moment of the complexity (see \cite{subag2017complexity} for an application to the pure spherical $p$-spin model) as well as perform heuristic replica calculations to obtain the quenched complexity (see for instance \cite{ros2019complex}). 
	\end{enumerate}
	
	\section{An application: annealed complexity of the pure spherical $p$-spin model}
	
	We essentially detail here a calculation performed by physicists, and made rigorous in \cite{auffinger2013random}. We follow here the derivation of this last paper. For (anterior) theoretical physics derivation of the complexity of similar models using the Kac-Rice formula along with heuristic theoretical physics arguments (giving nevertheless the exact result), one can for instance read the works of Fyodorov: \cite{fyodorov2004complexity}, \cite{fyodorov2007replica}. 
	
	\subsection{Statement of the problem}
	Consider $N \geq 1$, $p \geq 3$, and define the function $f_{N,p}$ on the unit sphere $\mathbb{S}^{N-1}$:
	\begin{align}
	f_{N,p}(\sigma) &\triangleq \sum_{1 \leq i_1,\cdots,i_p \leq N} J_{i_1,\cdots i_p} \sigma_{i_1} \cdots \sigma_{i_p},
	\end{align}
	in which $J_{i_1,\cdots i_p} \overset{i.i.d.}{\sim} \mathcal{N}(0,1)$. In physics terms, if $H_{N,p}$ is the Hamiltonian of the spherical $p$-spin model (defined on $\mathbb{S}^{N-1}(\sqrt{N})$), one has $f_{N,p}(\sigma) = \frac{1}{\sqrt{N}} H_{N,p}(\sqrt{N}\sigma)$.  For any Borel set $B \subseteq \bbR$, we want to compute the large $N$ limit of the expectation of $\mathrm{Crit}_{N,p}^0(B)$, the number of local minima $\sigma$ of $f_{N,p}$ such that $f_{N,p}(\sigma) \in \sqrt{N} B$. 
	A direct application of the Kac-Rice formula yields:
	\begin{align*}
	\EE \, \mathrm{Crit}_{N,p}^0(B) &= \int_{\mathbb{S}^{N-1}} \mu_N(\mathrm{d}\sigma) \varphi_{\mathrm{grad}\, f_{N,p}(\sigma)}(0) \, \EE \left[\left|\det \, \mathrm{Hess}\, f_{N,p}(\sigma)\right| \, \mathds{1}\left[f_{n,p}(\sigma) \in \sqrt{N} B,\, \mathrm{Hess}\, f_{N,p}(\sigma) \geq 0\right] \middle| \mathrm{grad}\, f_{N,p}(\sigma) = 0 \right],
	\end{align*}
in which $\mu_N$ is the usual surface measure on $\mathbb{S}^{N-1}$. Note that here $\mathrm{grad}$ and $\mathrm{Hess}$ denote the \emph{Riemannian} gradient and Hessian on the sphere, while we will denote $\nabla$, $\nabla^2$ the Euclidian gradient and Hessian. 
	
	\subsection{The distribution of $(f(\sigma),\mathrm{grad}\, f(\sigma),\mathrm{Hess}\, f(\sigma))$}
	
	We fix $\sigma  \in \mathbb{S}^{N-1}$. It is trivial to see that all three variables $(f(\sigma),\mathrm{grad}\, f(\sigma),\mathrm{Hess}\, f(\sigma))$ are Gaussian centered random variables. We thus simply need compute their correlations to characterize their joint distribution. We naturally identify the tangent space $\mathcal{T}_\sigma(\mathbb{S}^{N-1})$ with $\bbR^{N-1}$. If we denote $P_\sigma^\perp$ the orthogonal projector on $\{\sigma\}^\perp$, one has:
	\begin{align}
	\mathrm{grad}\, f(\sigma) &= P_\sigma^\perp \nabla f(\sigma), \\
	\mathrm{Hess}\, f(\sigma) &= P_\sigma^\perp \nabla^2 f(\sigma) P_{\sigma}^\perp - \braket{\sigma,\nabla f(\sigma)} P_\sigma^\perp.
	\end{align}
	
	So for instance:
	
	\begin{align*}
	\EE \left[\mathrm{grad}\, f(\sigma) \mathrm{grad}\, f(\sigma)^\intercal \right] &=  P_\sigma^\perp\EE \left[\nabla f(\sigma) \nabla f(\sigma)^\intercal \right] P_\sigma^\perp, \\
	&= p  P_\sigma^\perp.
	\end{align*}
	Uisng the same kind of calculation, one easily obtains the joint distribution:
	\begin{lemma}\label{lemma:joint}
	The joint law of $(f(\sigma),\mathrm{grad}\, f(\sigma),\mathrm{Hess}\, f(\sigma))$ is the following:
	\begin{align}
	\begin{cases}
	f(\sigma) &\overset{d}{=} Z \\
	\mathrm{grad}\, f(\sigma) &\overset{d}{=} \sqrt{p} \bm{g} \\
	\mathrm{Hess}\, f(\sigma) &\overset{d}{=} \sqrt{2 (N-1)p(p-1)} M_{N-1} - p Z \, \mathrm{Id}_{N-1}
	\end{cases}
	\end{align}
	in which $Z \sim \mathcal{N}(0,1)$, $\bm{g} \sim \mathcal{N}(0,\mathrm{Id}_{N-1})$ , and $M_{N-1}$ is a GOE matrix of size $(N-1)$ with the convention $\EE M_{ij}^2 = \frac{1+\delta_{ij}}{2(N-1)}$. The variables $(Z,\bm{g},M_{N-1})$ are pairwise independent.
	\end{lemma}
	
	Let us make the following remarks:
	\begin{enumerate}
	\item The distribution of all variables is independent of $\sigma$.
	\item The variables $(f,\mathrm{Hess}\, f)$ are independent from $\mathrm{grad}\,f$, so the conditioning in the Kac-Rice formula will be trivial.
	\item From the gradient distribution, one easily obtains:
	\begin{align*}
	 \varphi_{\mathrm{grad}\, f_{N,p}(\sigma)}(0) &= e^{- \frac{N-1}{2} \ln (2 \pi p)}.
	\end{align*}
	\end{enumerate}
	
	Recalling that the volume of the unit sphere is $V(\mathbb{S}^{N-1}) = 2 \pi^{N/2} / \Gamma(N/2)$, one deduces from the Kac-Rice formula:
	\begin{align}\label{eq:0}
	\EE \, \mathrm{Crit}_{N,p}^0(B) &= \frac{2 \pi^{N/2}}{\Gamma(N/2)}e^{- \frac{N-1}{2} \ln (2 \pi p)} \left[2 (N-1)p(p-1)\right]^{\frac{N-1}{2}} \EE \left[|\det H_{N-1}| \mathds{1}(H_{N-1} \geq 0, \, z \in \sqrt{N} B)\right],
	\end{align}
	in which $H_{N-1} \triangleq M_{N-1} - \sqrt{\frac{p}{2 (N-1)(p-1)}} z$. $z$ is a standard Gaussian variable and $M_{N-1}$ is a GOE matrix of size $N-1$ (see Lemma.~\ref{lemma:joint}). It is now completely clear that we reduced a random differential geometry problem (counting the number of critical points of a random smooth function) to a random matrix theory problem. 
	
	\subsection{Simplification of the problem}
	
	It is clear from Eq.~\ref{eq:0} that the following lemma will be useful:
	\begin{lemma}\label{lemma:simplification} Let $G \subseteq \bbR$ a Borel set, $X \sim \mathcal{N}(0,t^2)$ (for a $t > 0$) and $M_{N-1} \sim \mathrm{GOE}(N-1)$. Then:
	\begin{align}
	\EE &\left[|\det (M_{N-1} - X \mathrm{Id}_{N-1})| \mathds{1}( (M_{N-1} - X \mathrm{Id}_{N-1}) \geq 0, X \in G)\right] \nonumber \\
	&\qquad = \frac{\Gamma\left(\frac{N}{2}\right)(N-1)^{-\frac{N}{2}}}{\sqrt{\pi t^2}} \EE_{\mathrm{GOE}(N)} \left[e^{-\frac{N}{2}\left(\frac{1}{(N-1)t^2} - 1\right) \lambda_0^2} \mathds{1} \left(\lambda_0 \in \sqrt{\frac{N-1}{N}} G\right)\right].
	\end{align}
	In this equation, $\lambda_0$ is the smallest eigenvalue of a random matrix from the GOE$(N)$ ensemble. 
	\end{lemma}

%	We do not prove this lemma here, as it is proven as a particular case of Lemma~3.3. of \cite{auffinger2013random}, which is proven in the same paper. The idea of the proof is to write explicitly the joint law of the eigenvalues of $M_{N-1}$, denoted $\lambda_1 \leq \cdots \leq \lambda_{N-1}$, and to interpret $X$ as the smallest eigenvalue of a larger $\mathrm{GOE}(N)$ matrix with eigenvalues $X \leq \lambda_1 \leq \cdots \leq \lambda_{N-1}$, and use Selberg's integral.  Applying this lemma to Eq.~\ref{eq:0} with $t = \sqrt{\frac{p}{2 (N-1)(p-1)}}$ and  $G = (t\sqrt{N}) B$ yields the main result of the Kac-Rice formula at finite $N$:
	\begin{align}\label{eq:1}
	\EE \, \mathrm{Crit}_{N,p}^0(B) &= 2 \sqrt{\frac{2}{p}} (p-1)^{\frac{N}{2}} \EE_{\mathrm{GOE}(N)} \left[e^{-N \frac{p-2}{2p} \lambda_0^2} \mathds{1} \left(\lambda_0 \in  \sqrt{\frac{p}{2(p-1)}} B\right)\right]
\end{align}		
	
	
	\subsection{The large $N$ limit}
	
		
	We are interested in the annealed limit complexity	$\lim_{N \to \infty} \frac{1}{N} \ln \EE \, \mathrm{Crit}_{N,p}^0(B)$. It it thus clear from Eq.~\ref{eq:1} that we need to study the large deviations for the smallest eigenvalue of a $\mathrm{GOE}(N)$ matrix. We state this result in an informal way, see Theorem A.1 of \cite{auffinger2013random} for a more precise statement, and the LDP for eigenvalues of all finite index:
\begin{lemma}[Informal]\label{lemma:ldp}
Let $\lambda_0$ be the smallest eigenvalue of a $\mathrm{GOE}(N)$ matrix, and $I \subseteq \bbR$. Then
\begin{align}
\lim_{N \to \infty} \frac{1}{N} \ln \mathbb{P}\left[\lambda_0 \in I\right] &= - \sup_{x \in I} F(x),
\end{align}
with $F$ defined as:
\begin{align}
F(x) &\triangleq \begin{cases} \infty &\mathrm{if}\,x \geq -\sqrt{2}  \\
 \frac{1}{\sqrt{2}} \int_{\sqrt{2}}^{-x} \mathrm{d}z\sqrt{\frac{z^2}{2} -1} & \mathrm{otherwise}
\end{cases}.
\end{align}
\end{lemma}
Using Lemma.~\ref{lemma:ldp} alongside Eq.~\ref{eq:1} and Varadhan's lemma directly yields a final estimate for the annealed complexity:
\begin{align}\label{eq:2}
\lim_{N \to \infty} \frac{1}{N} \ln \EE \, \mathrm{Crit}_{N,p}^0(B) &= \frac{1}{2} \ln (p-1) + \sup_{x \in \sqrt{\frac{p}{2(p-1)}} B} \left[-\frac{p-2}{2p}x^2 - F(x)\right].
\end{align}
	
	\subsection{Description of the results}\label{subsec:summary}
	
	If one chooses in Eq.~\ref{eq:2}, $B = (-\infty,u)$ for $u \in \bbR$, one effectively counts the local minima of the $p$-spin of extensive energy smaller than $Nu$. In this case, the supremum in Eq.~\ref{eq:2} can be analytically performed, and one obtains an analytic form for $\lim_{N \to \infty} \frac{1}{N} \ln \EE \, \mathrm{Crit}_{N,p}^0((-\infty,u))$. On can perform the same calculation we did for fixed points of any fixed index $k \in \mathbb{N}$ (see again \cite{auffinger2013random}), and if we define:
	\begin{align}
	\Theta_k(u) \triangleq \lim_{N \to \infty} \frac{1}{N} \ln \EE \, \mathrm{Crit}_{N,p}^k((-\infty,u)),
	\end{align}
	there exists analytic expressions for all $\Theta_k(u)$ functions, see Eq.~2.16 of \cite{auffinger2013random}. Defining the \emph{threshold} energy $E_\infty \equiv 2 \sqrt{\frac{p-1}{p}}$, we can plot these functions, see Fig.~\ref{fig:fig}.
		\begin{figure}
		\centering
	\includegraphics[scale=0.4]{figure.pdf}
	\caption{The functions $\Theta_k$ for the first indices}\label{fig:fig}
	\end{figure}

	
	\paragraph{Additional remarks}
	\begin{enumerate}
\item The local minima always dominate the complexity for all energies below $-N E_\infty$, while for all energies above $-N E_\infty$, the complexity is dominated by critical points of diverging index.
\item One can perform similar calculations for these critical points whose indices diverge with $N$, see \cite{auffinger2013complexity}  for the rigorous derivation.
\item This rigorous calculation can also be generalized to the mixed spherical $p$-spin case, see again \cite{auffinger2013complexity}, which exhaust all stationary isotropic Gaussian random fields on the sphere by Schoenberg's theorem.
	\end{enumerate}
	
 \bibliographystyle{alpha}
\bibliography{refs}
\end{document}
	